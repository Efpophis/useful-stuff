% !TEX TS-program = pdflatex
% !TEX encoding = UTF-8 Unicode

% This is a simple template for a LaTeX document using the "article" class.
% See "book", "report", "letter" for other types of document.

\documentclass[11pt]{article} % use larger type; default would be 10pt

\usepackage[utf8]{inputenc} % set input encoding (not needed with XeLaTeX)

%%% Examples of Article customizations
% These packages are optional, depending whether you want the features they provide.
% See the LaTeX Companion or other references for full information.

%%% PAGE DIMENSIONS
\usepackage{geometry} % to change the page dimensions
\geometry{margin=1in} % or letterpaper (US) or a5paper or....
% \geometry{margin=2in} % for example, change the margins to 2 inches all round
% \geometry{landscape} % set up the page for landscape
%   read geometry.pdf for detailed page layout information

\usepackage{graphicx} % support the \includegraphics command and options

% \usepackage[parfill]{parskip} % Activate to begin paragraphs with an empty line rather than an indent

%%% PACKAGES
\usepackage{booktabs} % for much better looking tables
\usepackage{array} % for better arrays (eg matrices) in maths
\usepackage{paralist} % very flexible & customisable lists (eg. enumerate/itemize, etc.)
\usepackage{verbatim} % adds environment for commenting out blocks of text & for better verbatim
\usepackage{subfig} % make it possible to include more than one captioned figure/table in a single float
\usepackage{lastpage} % for \lastpage ref
\usepackage{color, colortbl}
\usepackage{xcolor}
\usepackage[normalem]{ulem}
\usepackage[hyphens]{url}
\usepackage[pdftex]{hyperref}
\usepackage{color, colortbl}
\usepackage{xcolor}
\usepackage{hhline}

\newcommand*\setfgcolor[1]{\gdef\fgcolor{#1}}

\definecolor{Gray}{gray}{0.5}
\definecolor{Black}{rgb}{0,0,0}
\definecolor{White}{rgb}{1,1,1}
% These packages are all incorporated in the memoir class to one degree or another...
\setfgcolor{Black}

\hypersetup{
	colorlinks	= true,	 %Colors links instead of ugly boxes
	urlcolor	= blue,  %Color for external hyperlinks
	linkcolor	= black, %Color of internal links
	citecolor	= red 	 %Color of citations
}

\providecommand{\tabularnewline}{\\}

%%% HEADERS & FOOTERS
\usepackage{fancyhdr} % This should be set AFTER setting up the page geometry
\pagestyle{fancy} % options: empty , plain , fancy
\fancypagestyle{mystyle}
{
	\newgeometry{margin=1in, bottom=1in, top=1in}
	\fancyhfoffset{0pt}
	\footskip 30.0pt
	\raggedright
	\renewcommand\headrulewidth{0pt}
	\fancyhead{}
	\fancyfoot{}
	\fancyfoot[L]{}
	\fancyfoot[C]{\footnotesize Page \thepage\ of \pageref{LastPage} }
	\fancyfoot[R]{}
}

%%% SECTION TITLE APPEARANCE
%%% END Article customizations

\setlength{\parindent}{0in}

%%% The "real" document content comes below...

\title{\LaTeX Document Template}
\author{Bill Crossley}
%\date{} % Activate to display a given date or no date (if empty),
         % otherwise the current date is printed 

\begin{document}

\begin{titlepage}
\vfill
\maketitle
\thispagestyle{empty}
\vfill
\end{titlepage}

\pagestyle{mystyle}

\tableofcontents{}

\pagebreak

\section{Overview}

Overview text goes here

\section{Some Section Title}

Some paragraph.\\
\bigskip
Last Paragraph.

\section{Some Other Section Title}

Section with other content.

\subsection{Sub-section title}

Sub-section content.

\subsubsection{Sub-Sub-Section title}

Let's not get TOO carried away here...

\paragraph{Interesting Paragraph Thing:}Content of interesting paragraph thing.

\pagebreak

\section{Nifty Examples}

\subsection{Itemized List}
This is how to do a bulleted list.

\begin{itemize}
\item One
\item Two
\item Five
\item Three, Sir!
\item Three!
\end{itemize}

\subsection{Numbered List}

\begin{enumerate}
\item some content
\item some more content
\item some content with a sub-list
	\begin{enumerate}
	\item sub thing
		\begin{enumerate}
		\item sub-sub thing
		\end{enumerate}
	\item sub thingy
	\end{enumerate}
\item neat, huh?
\end{enumerate}

\subsection{A Descriptive List}

The source for this section shows how to make something called a \emph{descriptive list}.

\begin{description}
\item [Italics]
This is how you make text appear in {\it italics}.  Another way that allows nesting: {\em This is in italics \em but this isn't \em and this is again}.
\item [Bold]
This is how you make a text appear in {\bf bold}.  {\bf \em bold \em and \em italics}? You betcha.
\item[Underline]
Want to \uline{underline} stuff?
	\begin{description}
	\item [Twice] \uuline{Twice}? 
	\item [Squiggly] \uwave{Squiggly}? 
	\item [Strikeout] \sout{Nope}.
	\item [Slashout] \xout{Removed}. 
	\item [Dashing] \dashuline{Dashing}! 
	\item [Dotty] \dotuline{Dotty}.
	\end{description}
\end{description}

Of course, different list types can be nested within each other, too.

\subsection{Table}

Here's how to make a nice table.\\
\bigskip
% table format: first colum is left aligned, 2nd colum is centered, 3rd is paragraph aligned, fixed size
\begin{tabular}{|l|c|p{11cm}|}
\hline
% column headings
\rowcolor{Gray} Col 1 & Col 2 & Col 3\tabularnewline
\hline
Col 1 text & Col 2 text & The text for colum number 3, first row.\tabularnewline
\hline
Row 2, Col 1 & Row 2, Col 2 & Col 3 text for row 2 - see where this is going?
\tabularnewline
\hline
\end{tabular}


\end{document}
